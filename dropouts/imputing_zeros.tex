\documentclass[11pt]{article}

\usepackage{times,fullpage}

\title{Identifying dropouts in Drop-seq data}

\author{Guilherme de Sena Brandine \and Chao Deng \and Andrew D. Smith}

\begin{document}

\maketitle

\section{Introduction}

Singe-cell RNA-Seq is a technique that allows transcriptome proles
of tissues in a single-cell resolution. It has a wide variety of
applications that range from analysis of tissue heterogeneity in
complex systems like tumors and blood tissues to temporal
reconstruction of difrentiation processes like embryogenesis. However,
due to the small amount of starting material that is the set of
messenger RNAs in a single cell, data generated by this procedure
suers from large amounts of technical noise.  One particular
obstacle in analyzing such data are dropout events, in which the
messenger RNA fails to be converted into cDNA, and as such it is not
amplied and in all posterior steps of analysis the gene's count will
be set to zero, whereas in truth it is actually expressed. Since many
genes in a single cell are not expressed for other biological reasons
(eg, dierent stages of cell cycle or genes that simply aren't
relevant to the cell's speciality), it is dicult to decide the
actual origin of a zero.

\section{Results}

\subsection{Model assumptions}
\label{sec:assumptions}

We assume a gene expression profile is a vector of expression values,
all of which sum to 1.

\subsection{Complete data likelihood}

In this section we give the complete data likelhood for the model
outlined above.

\subsection{Adapting the model to be more biological meaningful}

Here we elaborate on the framework given in
Section~\ref{sec:assumptions}. In particular,
\begin{itemize}
\item We move away from the assumption that dropouts happen uniformly
  between genes. We will allow gene-specific prior probabilities on
  dropouts.
\item We assumed that a set of neighbors were known for each cell.
  Now we will use a set of other cells, and their contributions will
  be weighed based on their distance to the cell of interest.
\end{itemize}

\end{document}
