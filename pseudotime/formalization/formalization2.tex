\documentclass[11pt]{article}

\usepackage{fullpage,hyperref,hypcap,namedplus,amsmath,amssymb,enumitem}
\hypersetup{colorlinks=true,allcolors=black}

\title{\bf Modeling gene expression temporal variation}
\author{\bf Guilherme de Sena Brandine}
\date{\today}

\begin{document}

\maketitle

\section{Introduction}
Define $T = \{ t_1, \dots t_n\} \subset [0,1]$ a set of time points in which we have samples of the gene values in that time, that is, for each time point $t_j$ we have a corresponding set $X_j = \{x_{j1} \dots x_{jn} \}$ of observations for this particular time. The goal is to find a function $\mu(t)$ that best characterizes the temporal evolution of the gene values. 

\section{Previous definitions of expression variation with pseudotime}
Most methods rely on \textbf{generalized additive models} that correlate the univariate response (here, the pseudotime) with the gene expression values. Formally:
$$
\mathbb{E}(t) = \alpha_0 + \sum_{i=1}^{D} f_i (x_i)
$$

Where $\alpha_0$ is a constant, $D$ is the number of genes, $X_i$ is the expression value for gene $i$ at some time point and $f_i$ is a smooth function with some assumptions. After the fitting is done through the backfitting algorithm, each function $f_i$ is said to represent how the time varies with gene expression, whence its inverse describes the other way around. In more detail, here's how each paper approaches the GAM fitting:
\\
 - Monocle and TSCAN: Although they clame to use GAM for their genes, they do not explicitly specify their assumptions on the function $f$ other than the fact that they're smooth.\\
 \\
 - Wanderlust: Does not do GAM fitting. Once they have the order of the cells, the expression is given by the median over a sliding window of the adjacent cells. \\
\\
 - DPT: Uses a two-part GLM (modified Hurdle model) that allows to quantify both the proportion of cells expressing a gene and the mean expression level. Let $Z_{ig}$ be a Bernoulli indicating whether gene $g$ is expressed in cell $i$ and $Y_{ig}$ be the expression value, then:
 
 $$
 \text{logit} (P(Z_{ig} = 1)) = X_i \beta_{g}^{D}
 $$
 $$
 P(Y_{ig} = y | Z_{ig} = 1) = \mathcal{N} (X_i \beta_{g}^{C}, \sigma^{2}_{g})
 $$
 
 Where $X_i$ is the data and $\beta_g^{C}$ and $\beta_g^{D}$ are the continuous and discrete regression components, respectively. Thus, note that the fitted function is piecewise linear. 

\section{Modeling the behavior of $\mu$ and $\sigma$}
We define $\mu(t)$ as a family of functions that simulate how genes can biologically behave throughout time. We define three types of functions:\\
\\
 - Oscillatory genes (eg, cell cycle genes), given by: $f_{o} (t) = \alpha_{o} \sin (\beta_{o} t + \delta_{o})$ for some parameters $\alpha_{o}, \beta_{o}, \delta_{o}$. \\
 - Continuously increasing/decreasing genes, herein given by: $f_{id} (t) = \alpha_{id} x^{\beta_{id}} + \delta_{id}$ (note that this function can be constant if $\alpha_{id} = 0$)\\
 - Swich genes modelled by heaveside functions: $f_{s} (t) = \mathbb{I}_{t > \alpha_{s}}$, which can also be used to model transitions between oscillation/increase-decrease behaviors: \\
 \\
 We model $\mu(t)$ as a combination of these parameters:
 
 $$
 \mu (t) = f_o (t) f_s (t) + f_{id} (t) (1 - f_s(t))
 $$
We can model the samples as either a normal distribution (for the case where normalization gives continuous values, such as RPKM or DESeq), or a Negative Binomial (for the discrete case):
$$
x_{ji} \sim 
\begin{cases}
\mathcal{N} (\mu(t_j), \sigma^2(t_j)) \text{ if x is continuous } \\
NB (\mu(t_j), \sigma^2(t_j)) \text{ if x is discrete}\\
\end{cases}
$$

Where $\sigma^2 (t) = \mu(t) + \epsilon \mu^2 (t)$ is the well-characterized overdispersion that represents technical and biological noise in gene expression reads.\\
\\
The values of $\hat{\mu}(t_j)$ can thus be estimated by the UMVUEs of the Normal/NB distributions (both are the same):
$$
\hat{\mu} (t_j) = \frac{\mathop{\sum}_{x \in X_j} x}{n}
$$
$$
\hat{\sigma ^2 }(t_j) = \frac{\sum_{x \in X_j} (x - \hat{\mu}(t))^2}{n-1}
$$
\section{Fitting the gene function}
To predict the behavior of $\mu(t)$ all we need to do is find the parameters $S = \{\alpha_{o}, \beta_{o}, \delta_{o}, \alpha_{id}, \beta_{id}, \delta_{id}, \alpha_{s}\}$ that minimize the squared error :
$$
E(S) = \sum_{i = 1}^{n} (\hat{\mu}(t_j) - \mu(t_j))^2 + (\hat{\sigma^2}(t_j) - \sigma^2 (t_j))^2
$$

Which is a simple convex function optimization problem. 
\end{document}
